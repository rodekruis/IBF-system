% A latex document created by ipypublish
% outline: ipypublish.templates.outline_schemas\latex_outline.latex.j2
% with segments:
% - standard-standard_packages: with standard nbconvert packages
% - standard-standard_definitions: with standard nbconvert definitions
% - ipypublish-doc_article: with the main ipypublish article setup
% - ipypublish-front_pages: with the main ipypublish title and contents page setup
% - ipypublish-biblio_natbib: with the main ipypublish bibliography
% - ipypublish-contents_output: with the main ipypublish content
% - ipypublish-contents_framed_code: with the input code wrapped and framed
% - ipypublish-glossary: with the main ipypublish glossary
%
%%%%%%%%%%%% DOCCLASS

\documentclass[10pt,parskip=half,
toc=sectionentrywithdots,
bibliography=totocnumbered,
captions=tableheading,numbers=noendperiod]{scrartcl}

%%%%%%%%%%%%

%%%%%%%%%%%% PACKAGES

\usepackage[T1]{fontenc} % Nicer default font (+ math font) than Computer Modern for most use cases
\usepackage{mathpazo}
\usepackage{graphicx}
\usepackage[skip=3pt]{caption}
\usepackage{adjustbox} % Used to constrain images to a maximum size
\usepackage[table]{xcolor} % Allow colors to be defined
\usepackage{enumerate} % Needed for markdown enumerations to work
\usepackage{amsmath} % Equations
\usepackage{amssymb} % Equations
\usepackage{textcomp} % defines textquotesingle
% Hack from http://tex.stackexchange.com/a/47451/13684:
\AtBeginDocument{%
    \def\PYZsq{\textquotesingle}% Upright quotes in Pygmentized code
}
\usepackage{upquote} % Upright quotes for verbatim code
\usepackage{eurosym} % defines \euro
\usepackage[mathletters]{ucs} % Extended unicode (utf-8) support
\usepackage[utf8x]{inputenc} % Allow utf-8 characters in the tex document
\usepackage{fancyvrb} % verbatim replacement that allows latex
\usepackage{grffile} % extends the file name processing of package graphics
                        % to support a larger range
% The hyperref package gives us a pdf with properly built
% internal navigation ('pdf bookmarks' for the table of contents,
% internal cross-reference links, web links for URLs, etc.)
\usepackage{hyperref}
\usepackage{longtable} % longtable support required by pandoc >1.10
\usepackage{booktabs}  % table support for pandoc > 1.12.2
\usepackage[inline]{enumitem} % IRkernel/repr support (it uses the enumerate* environment)
\usepackage[normalem]{ulem} % ulem is needed to support strikethroughs (\sout)
                            % normalem makes italics be italics, not underlines

\usepackage{translations}
\usepackage{microtype} % improves the spacing between words and letters
\usepackage{placeins} % placement of figures
% could use \usepackage[section]{placeins} but placing in subsection in command section
% Places the float at precisely the location in the LaTeX code (with H)
\usepackage{float}
\usepackage[colorinlistoftodos,obeyFinal,textwidth=.8in]{todonotes} % to mark to-dos
% number figures, tables and equations by section
% fix for new versions of texlive (see https://tex.stackexchange.com/a/425603/107738)
\let\counterwithout\relax
\let\counterwithin\relax
\usepackage{chngcntr}
% header/footer
\usepackage[footsepline=0.25pt]{scrlayer-scrpage}

% bibliography formatting
\usepackage[numbers, square, super, sort&compress]{natbib}
% hyperlink doi's
\usepackage{doi}

    % define a code float
    \usepackage{newfloat} % to define a new float types
    \DeclareFloatingEnvironment[
        fileext=frm,placement={!ht},
        within=section,name=Code]{codecell}
    \DeclareFloatingEnvironment[
        fileext=frm,placement={!ht},
        within=section,name=Text]{textcell}
    \DeclareFloatingEnvironment[
        fileext=frm,placement={!ht},
        within=section,name=Text]{errorcell}

    \usepackage{listings} % a package for wrapping code in a box
    \usepackage[framemethod=tikz]{mdframed} % to fram code

%%%%%%%%%%%%

%%%%%%%%%%%% DEFINITIONS

% Pygments definitions

\makeatletter
\def\PY@reset{\let\PY@it=\relax \let\PY@bf=\relax%
    \let\PY@ul=\relax \let\PY@tc=\relax%
    \let\PY@bc=\relax \let\PY@ff=\relax}
\def\PY@tok#1{\csname PY@tok@#1\endcsname}
\def\PY@toks#1+{\ifx\relax#1\empty\else%
    \PY@tok{#1}\expandafter\PY@toks\fi}
\def\PY@do#1{\PY@bc{\PY@tc{\PY@ul{%
    \PY@it{\PY@bf{\PY@ff{#1}}}}}}}
\def\PY#1#2{\PY@reset\PY@toks#1+\relax+\PY@do{#2}}

\expandafter\def\csname PY@tok@w\endcsname{\def\PY@tc##1{\textcolor[rgb]{0.73,0.73,0.73}{##1}}}
\expandafter\def\csname PY@tok@c\endcsname{\let\PY@it=\textit\def\PY@tc##1{\textcolor[rgb]{0.25,0.50,0.50}{##1}}}
\expandafter\def\csname PY@tok@cp\endcsname{\def\PY@tc##1{\textcolor[rgb]{0.74,0.48,0.00}{##1}}}
\expandafter\def\csname PY@tok@k\endcsname{\let\PY@bf=\textbf\def\PY@tc##1{\textcolor[rgb]{0.00,0.50,0.00}{##1}}}
\expandafter\def\csname PY@tok@kp\endcsname{\def\PY@tc##1{\textcolor[rgb]{0.00,0.50,0.00}{##1}}}
\expandafter\def\csname PY@tok@kt\endcsname{\def\PY@tc##1{\textcolor[rgb]{0.69,0.00,0.25}{##1}}}
\expandafter\def\csname PY@tok@o\endcsname{\def\PY@tc##1{\textcolor[rgb]{0.40,0.40,0.40}{##1}}}
\expandafter\def\csname PY@tok@ow\endcsname{\let\PY@bf=\textbf\def\PY@tc##1{\textcolor[rgb]{0.67,0.13,1.00}{##1}}}
\expandafter\def\csname PY@tok@nb\endcsname{\def\PY@tc##1{\textcolor[rgb]{0.00,0.50,0.00}{##1}}}
\expandafter\def\csname PY@tok@nf\endcsname{\def\PY@tc##1{\textcolor[rgb]{0.00,0.00,1.00}{##1}}}
\expandafter\def\csname PY@tok@nc\endcsname{\let\PY@bf=\textbf\def\PY@tc##1{\textcolor[rgb]{0.00,0.00,1.00}{##1}}}
\expandafter\def\csname PY@tok@nn\endcsname{\let\PY@bf=\textbf\def\PY@tc##1{\textcolor[rgb]{0.00,0.00,1.00}{##1}}}
\expandafter\def\csname PY@tok@ne\endcsname{\let\PY@bf=\textbf\def\PY@tc##1{\textcolor[rgb]{0.82,0.25,0.23}{##1}}}
\expandafter\def\csname PY@tok@nv\endcsname{\def\PY@tc##1{\textcolor[rgb]{0.10,0.09,0.49}{##1}}}
\expandafter\def\csname PY@tok@no\endcsname{\def\PY@tc##1{\textcolor[rgb]{0.53,0.00,0.00}{##1}}}
\expandafter\def\csname PY@tok@nl\endcsname{\def\PY@tc##1{\textcolor[rgb]{0.63,0.63,0.00}{##1}}}
\expandafter\def\csname PY@tok@ni\endcsname{\let\PY@bf=\textbf\def\PY@tc##1{\textcolor[rgb]{0.60,0.60,0.60}{##1}}}
\expandafter\def\csname PY@tok@na\endcsname{\def\PY@tc##1{\textcolor[rgb]{0.49,0.56,0.16}{##1}}}
\expandafter\def\csname PY@tok@nt\endcsname{\let\PY@bf=\textbf\def\PY@tc##1{\textcolor[rgb]{0.00,0.50,0.00}{##1}}}
\expandafter\def\csname PY@tok@nd\endcsname{\def\PY@tc##1{\textcolor[rgb]{0.67,0.13,1.00}{##1}}}
\expandafter\def\csname PY@tok@s\endcsname{\def\PY@tc##1{\textcolor[rgb]{0.73,0.13,0.13}{##1}}}
\expandafter\def\csname PY@tok@sd\endcsname{\let\PY@it=\textit\def\PY@tc##1{\textcolor[rgb]{0.73,0.13,0.13}{##1}}}
\expandafter\def\csname PY@tok@si\endcsname{\let\PY@bf=\textbf\def\PY@tc##1{\textcolor[rgb]{0.73,0.40,0.53}{##1}}}
\expandafter\def\csname PY@tok@se\endcsname{\let\PY@bf=\textbf\def\PY@tc##1{\textcolor[rgb]{0.73,0.40,0.13}{##1}}}
\expandafter\def\csname PY@tok@sr\endcsname{\def\PY@tc##1{\textcolor[rgb]{0.73,0.40,0.53}{##1}}}
\expandafter\def\csname PY@tok@ss\endcsname{\def\PY@tc##1{\textcolor[rgb]{0.10,0.09,0.49}{##1}}}
\expandafter\def\csname PY@tok@sx\endcsname{\def\PY@tc##1{\textcolor[rgb]{0.00,0.50,0.00}{##1}}}
\expandafter\def\csname PY@tok@m\endcsname{\def\PY@tc##1{\textcolor[rgb]{0.40,0.40,0.40}{##1}}}
\expandafter\def\csname PY@tok@gh\endcsname{\let\PY@bf=\textbf\def\PY@tc##1{\textcolor[rgb]{0.00,0.00,0.50}{##1}}}
\expandafter\def\csname PY@tok@gu\endcsname{\let\PY@bf=\textbf\def\PY@tc##1{\textcolor[rgb]{0.50,0.00,0.50}{##1}}}
\expandafter\def\csname PY@tok@gd\endcsname{\def\PY@tc##1{\textcolor[rgb]{0.63,0.00,0.00}{##1}}}
\expandafter\def\csname PY@tok@gi\endcsname{\def\PY@tc##1{\textcolor[rgb]{0.00,0.63,0.00}{##1}}}
\expandafter\def\csname PY@tok@gr\endcsname{\def\PY@tc##1{\textcolor[rgb]{1.00,0.00,0.00}{##1}}}
\expandafter\def\csname PY@tok@ge\endcsname{\let\PY@it=\textit}
\expandafter\def\csname PY@tok@gs\endcsname{\let\PY@bf=\textbf}
\expandafter\def\csname PY@tok@gp\endcsname{\let\PY@bf=\textbf\def\PY@tc##1{\textcolor[rgb]{0.00,0.00,0.50}{##1}}}
\expandafter\def\csname PY@tok@go\endcsname{\def\PY@tc##1{\textcolor[rgb]{0.53,0.53,0.53}{##1}}}
\expandafter\def\csname PY@tok@gt\endcsname{\def\PY@tc##1{\textcolor[rgb]{0.00,0.27,0.87}{##1}}}
\expandafter\def\csname PY@tok@err\endcsname{\def\PY@bc##1{\setlength{\fboxsep}{0pt}\fcolorbox[rgb]{1.00,0.00,0.00}{1,1,1}{\strut ##1}}}
\expandafter\def\csname PY@tok@kc\endcsname{\let\PY@bf=\textbf\def\PY@tc##1{\textcolor[rgb]{0.00,0.50,0.00}{##1}}}
\expandafter\def\csname PY@tok@kd\endcsname{\let\PY@bf=\textbf\def\PY@tc##1{\textcolor[rgb]{0.00,0.50,0.00}{##1}}}
\expandafter\def\csname PY@tok@kn\endcsname{\let\PY@bf=\textbf\def\PY@tc##1{\textcolor[rgb]{0.00,0.50,0.00}{##1}}}
\expandafter\def\csname PY@tok@kr\endcsname{\let\PY@bf=\textbf\def\PY@tc##1{\textcolor[rgb]{0.00,0.50,0.00}{##1}}}
\expandafter\def\csname PY@tok@bp\endcsname{\def\PY@tc##1{\textcolor[rgb]{0.00,0.50,0.00}{##1}}}
\expandafter\def\csname PY@tok@fm\endcsname{\def\PY@tc##1{\textcolor[rgb]{0.00,0.00,1.00}{##1}}}
\expandafter\def\csname PY@tok@vc\endcsname{\def\PY@tc##1{\textcolor[rgb]{0.10,0.09,0.49}{##1}}}
\expandafter\def\csname PY@tok@vg\endcsname{\def\PY@tc##1{\textcolor[rgb]{0.10,0.09,0.49}{##1}}}
\expandafter\def\csname PY@tok@vi\endcsname{\def\PY@tc##1{\textcolor[rgb]{0.10,0.09,0.49}{##1}}}
\expandafter\def\csname PY@tok@vm\endcsname{\def\PY@tc##1{\textcolor[rgb]{0.10,0.09,0.49}{##1}}}
\expandafter\def\csname PY@tok@sa\endcsname{\def\PY@tc##1{\textcolor[rgb]{0.73,0.13,0.13}{##1}}}
\expandafter\def\csname PY@tok@sb\endcsname{\def\PY@tc##1{\textcolor[rgb]{0.73,0.13,0.13}{##1}}}
\expandafter\def\csname PY@tok@sc\endcsname{\def\PY@tc##1{\textcolor[rgb]{0.73,0.13,0.13}{##1}}}
\expandafter\def\csname PY@tok@dl\endcsname{\def\PY@tc##1{\textcolor[rgb]{0.73,0.13,0.13}{##1}}}
\expandafter\def\csname PY@tok@s2\endcsname{\def\PY@tc##1{\textcolor[rgb]{0.73,0.13,0.13}{##1}}}
\expandafter\def\csname PY@tok@sh\endcsname{\def\PY@tc##1{\textcolor[rgb]{0.73,0.13,0.13}{##1}}}
\expandafter\def\csname PY@tok@s1\endcsname{\def\PY@tc##1{\textcolor[rgb]{0.73,0.13,0.13}{##1}}}
\expandafter\def\csname PY@tok@mb\endcsname{\def\PY@tc##1{\textcolor[rgb]{0.40,0.40,0.40}{##1}}}
\expandafter\def\csname PY@tok@mf\endcsname{\def\PY@tc##1{\textcolor[rgb]{0.40,0.40,0.40}{##1}}}
\expandafter\def\csname PY@tok@mh\endcsname{\def\PY@tc##1{\textcolor[rgb]{0.40,0.40,0.40}{##1}}}
\expandafter\def\csname PY@tok@mi\endcsname{\def\PY@tc##1{\textcolor[rgb]{0.40,0.40,0.40}{##1}}}
\expandafter\def\csname PY@tok@il\endcsname{\def\PY@tc##1{\textcolor[rgb]{0.40,0.40,0.40}{##1}}}
\expandafter\def\csname PY@tok@mo\endcsname{\def\PY@tc##1{\textcolor[rgb]{0.40,0.40,0.40}{##1}}}
\expandafter\def\csname PY@tok@ch\endcsname{\let\PY@it=\textit\def\PY@tc##1{\textcolor[rgb]{0.25,0.50,0.50}{##1}}}
\expandafter\def\csname PY@tok@cm\endcsname{\let\PY@it=\textit\def\PY@tc##1{\textcolor[rgb]{0.25,0.50,0.50}{##1}}}
\expandafter\def\csname PY@tok@cpf\endcsname{\let\PY@it=\textit\def\PY@tc##1{\textcolor[rgb]{0.25,0.50,0.50}{##1}}}
\expandafter\def\csname PY@tok@c1\endcsname{\let\PY@it=\textit\def\PY@tc##1{\textcolor[rgb]{0.25,0.50,0.50}{##1}}}
\expandafter\def\csname PY@tok@cs\endcsname{\let\PY@it=\textit\def\PY@tc##1{\textcolor[rgb]{0.25,0.50,0.50}{##1}}}

\def\PYZbs{\char`\\}
\def\PYZus{\char`\_}
\def\PYZob{\char`\{}
\def\PYZcb{\char`\}}
\def\PYZca{\char`\^}
\def\PYZam{\char`\&}
\def\PYZlt{\char`\<}
\def\PYZgt{\char`\>}
\def\PYZsh{\char`\#}
\def\PYZpc{\char`\%}
\def\PYZdl{\char`\$}
\def\PYZhy{\char`\-}
\def\PYZsq{\char`\'}
\def\PYZdq{\char`\"}
\def\PYZti{\char`\~}
% for compatibility with earlier versions
\def\PYZat{@}
\def\PYZlb{[}
\def\PYZrb{]}
\makeatother

% ANSI colors
\definecolor{ansi-black}{HTML}{3E424D}
\definecolor{ansi-black-intense}{HTML}{282C36}
\definecolor{ansi-red}{HTML}{E75C58}
\definecolor{ansi-red-intense}{HTML}{B22B31}
\definecolor{ansi-green}{HTML}{00A250}
\definecolor{ansi-green-intense}{HTML}{007427}
\definecolor{ansi-yellow}{HTML}{DDB62B}
\definecolor{ansi-yellow-intense}{HTML}{B27D12}
\definecolor{ansi-blue}{HTML}{208FFB}
\definecolor{ansi-blue-intense}{HTML}{0065CA}
\definecolor{ansi-magenta}{HTML}{D160C4}
\definecolor{ansi-magenta-intense}{HTML}{A03196}
\definecolor{ansi-cyan}{HTML}{60C6C8}
\definecolor{ansi-cyan-intense}{HTML}{258F8F}
\definecolor{ansi-white}{HTML}{C5C1B4}
\definecolor{ansi-white-intense}{HTML}{A1A6B2}

% commands and environments needed by pandoc snippets
% extracted from the output of `pandoc -s`
\providecommand{\tightlist}{%
  \setlength{\itemsep}{0pt}\setlength{\parskip}{0pt}}
\DefineVerbatimEnvironment{Highlighting}{Verbatim}{commandchars=\\\{\}}
% Add ',fontsize=\small' for more characters per line
\newenvironment{Shaded}{}{}
\newcommand{\KeywordTok}[1]{\textcolor[rgb]{0.00,0.44,0.13}{\textbf{{#1}}}}
\newcommand{\DataTypeTok}[1]{\textcolor[rgb]{0.56,0.13,0.00}{{#1}}}
\newcommand{\DecValTok}[1]{\textcolor[rgb]{0.25,0.63,0.44}{{#1}}}
\newcommand{\BaseNTok}[1]{\textcolor[rgb]{0.25,0.63,0.44}{{#1}}}
\newcommand{\FloatTok}[1]{\textcolor[rgb]{0.25,0.63,0.44}{{#1}}}
\newcommand{\CharTok}[1]{\textcolor[rgb]{0.25,0.44,0.63}{{#1}}}
\newcommand{\StringTok}[1]{\textcolor[rgb]{0.25,0.44,0.63}{{#1}}}
\newcommand{\CommentTok}[1]{\textcolor[rgb]{0.38,0.63,0.69}{\textit{{#1}}}}
\newcommand{\OtherTok}[1]{\textcolor[rgb]{0.00,0.44,0.13}{{#1}}}
\newcommand{\AlertTok}[1]{\textcolor[rgb]{1.00,0.00,0.00}{\textbf{{#1}}}}
\newcommand{\FunctionTok}[1]{\textcolor[rgb]{0.02,0.16,0.49}{{#1}}}
\newcommand{\RegionMarkerTok}[1]{{#1}}
\newcommand{\ErrorTok}[1]{\textcolor[rgb]{1.00,0.00,0.00}{\textbf{{#1}}}}
\newcommand{\NormalTok}[1]{{#1}}

% Additional commands for more recent versions of Pandoc
\newcommand{\ConstantTok}[1]{\textcolor[rgb]{0.53,0.00,0.00}{{#1}}}
\newcommand{\SpecialCharTok}[1]{\textcolor[rgb]{0.25,0.44,0.63}{{#1}}}
\newcommand{\VerbatimStringTok}[1]{\textcolor[rgb]{0.25,0.44,0.63}{{#1}}}
\newcommand{\SpecialStringTok}[1]{\textcolor[rgb]{0.73,0.40,0.53}{{#1}}}
\newcommand{\ImportTok}[1]{{#1}}
\newcommand{\DocumentationTok}[1]{\textcolor[rgb]{0.73,0.13,0.13}{\textit{{#1}}}}
\newcommand{\AnnotationTok}[1]{\textcolor[rgb]{0.38,0.63,0.69}{\textbf{\textit{{#1}}}}}
\newcommand{\CommentVarTok}[1]{\textcolor[rgb]{0.38,0.63,0.69}{\textbf{\textit{{#1}}}}}
\newcommand{\VariableTok}[1]{\textcolor[rgb]{0.10,0.09,0.49}{{#1}}}
\newcommand{\ControlFlowTok}[1]{\textcolor[rgb]{0.00,0.44,0.13}{\textbf{{#1}}}}
\newcommand{\OperatorTok}[1]{\textcolor[rgb]{0.40,0.40,0.40}{{#1}}}
\newcommand{\BuiltInTok}[1]{{#1}}
\newcommand{\ExtensionTok}[1]{{#1}}
\newcommand{\PreprocessorTok}[1]{\textcolor[rgb]{0.74,0.48,0.00}{{#1}}}
\newcommand{\AttributeTok}[1]{\textcolor[rgb]{0.49,0.56,0.16}{{#1}}}
\newcommand{\InformationTok}[1]{\textcolor[rgb]{0.38,0.63,0.69}{\textbf{\textit{{#1}}}}}
\newcommand{\WarningTok}[1]{\textcolor[rgb]{0.38,0.63,0.69}{\textbf{\textit{{#1}}}}}

% Define a nice break command that doesn't care if a line doesn't already
% exist.
\def\br{\hspace*{\fill} \\* }

% Math Jax compatability definitions
\def\gt{>}
\def\lt{<}

\setcounter{secnumdepth}{5}

% Colors for the hyperref package
\definecolor{urlcolor}{rgb}{0,.145,.698}
\definecolor{linkcolor}{rgb}{.71,0.21,0.01}
\definecolor{citecolor}{rgb}{.12,.54,.11}

\DeclareTranslationFallback{Author}{Author}
\DeclareTranslation{Portuges}{Author}{Autor}

\DeclareTranslationFallback{List of Codes}{List of Codes}
\DeclareTranslation{Catalan}{List of Codes}{Llista de Codis}
\DeclareTranslation{Danish}{List of Codes}{Liste over Koder}
\DeclareTranslation{German}{List of Codes}{Liste der Codes}
\DeclareTranslation{Spanish}{List of Codes}{Lista de C\'{o}digos}
\DeclareTranslation{French}{List of Codes}{Liste des Codes}
\DeclareTranslation{Italian}{List of Codes}{Elenco dei Codici}
\DeclareTranslation{Dutch}{List of Codes}{Lijst van Codes}
\DeclareTranslation{Portuges}{List of Codes}{Lista de C\'{o}digos}

\DeclareTranslationFallback{Supervisors}{Supervisors}
\DeclareTranslation{Catalan}{Supervisors}{Supervisors}
\DeclareTranslation{Danish}{Supervisors}{Vejledere}
\DeclareTranslation{German}{Supervisors}{Vorgesetzten}
\DeclareTranslation{Spanish}{Supervisors}{Supervisores}
\DeclareTranslation{French}{Supervisors}{Superviseurs}
\DeclareTranslation{Italian}{Supervisors}{Le autorit\`{a} di vigilanza}
\DeclareTranslation{Dutch}{Supervisors}{supervisors}
\DeclareTranslation{Portuguese}{Supervisors}{Supervisores}

\definecolor{codegreen}{rgb}{0,0.6,0}
\definecolor{codegray}{rgb}{0.5,0.5,0.5}
\definecolor{codepurple}{rgb}{0.58,0,0.82}
\definecolor{backcolour}{rgb}{0.95,0.95,0.95}

\lstdefinestyle{mystyle}{
    commentstyle=\color{codegreen},
    keywordstyle=\color{magenta},
    numberstyle=\tiny\color{codegray},
    stringstyle=\color{codepurple},
    basicstyle=\ttfamily,
    breakatwhitespace=false,
    keepspaces=true,
    numbers=left,
    numbersep=10pt,
    showspaces=false,
    showstringspaces=false,
    showtabs=false,
    tabsize=2,
    breaklines=true,
    literate={\-}{}{0\discretionary{-}{}{-}},
  postbreak=\mbox{\textcolor{red}{$\hookrightarrow$}\space},
}

\lstset{style=mystyle}

\surroundwithmdframed[
  hidealllines=true,
  backgroundcolor=backcolour,
  innerleftmargin=0pt,
  innerrightmargin=0pt,
  innertopmargin=0pt,
  innerbottommargin=0pt]{lstlisting}

%%%%%%%%%%%%

%%%%%%%%%%%% MARGINS

 % Used to adjust the document margins
\usepackage{geometry}
\geometry{tmargin=1in,bmargin=1in,lmargin=1in,rmargin=1in,
nohead,includefoot,footskip=25pt}
% you can use showframe option to check the margins visually
%%%%%%%%%%%%

%%%%%%%%%%%% COMMANDS

% ensure new section starts on new page
\addtokomafont{section}{\clearpage}

% Prevent overflowing lines due to hard-to-break entities
\sloppy

% Setup hyperref package
\hypersetup{
    breaklinks=true,  % so long urls are correctly broken across lines
    colorlinks=true,
    urlcolor=urlcolor,
    linkcolor=linkcolor,
    citecolor=citecolor,
    }

% ensure figures are placed within subsections
\makeatletter
\AtBeginDocument{%
    \expandafter\renewcommand\expandafter\subsection\expandafter
    {\expandafter\@fb@secFB\subsection}%
    \newcommand\@fb@secFB{\FloatBarrier
    \gdef\@fb@afterHHook{\@fb@topbarrier \gdef\@fb@afterHHook{}}}%
    \g@addto@macro\@afterheading{\@fb@afterHHook}%
    \gdef\@fb@afterHHook{}%
}
\makeatother

% number figures, tables and equations by section
\counterwithout{figure}{section}
\counterwithout{table}{section}
\counterwithout{equation}{section}
\makeatletter
\@addtoreset{table}{section}
\@addtoreset{figure}{section}
\@addtoreset{equation}{section}
\makeatother
\renewcommand\thetable{\thesection.\arabic{table}}
\renewcommand\thefigure{\thesection.\arabic{figure}}
\renewcommand\theequation{\thesection.\arabic{equation}}

    % set global options for float placement
    \makeatletter
        \providecommand*\setfloatlocations[2]{\@namedef{fps@#1}{#2}}
    \makeatother

% align captions to left (indented)
\captionsetup{justification=raggedright,
singlelinecheck=false,format=hang,labelfont={it,bf}}

% shift footer down so space between separation line
\ModifyLayer[addvoffset=.6ex]{scrheadings.foot.odd}
\ModifyLayer[addvoffset=.6ex]{scrheadings.foot.even}
\ModifyLayer[addvoffset=.6ex]{scrheadings.foot.oneside}
\ModifyLayer[addvoffset=.6ex]{plain.scrheadings.foot.odd}
\ModifyLayer[addvoffset=.6ex]{plain.scrheadings.foot.even}
\ModifyLayer[addvoffset=.6ex]{plain.scrheadings.foot.oneside}
\pagestyle{scrheadings}
\clearscrheadfoot{}
\ifoot{\leftmark}
\renewcommand{\sectionmark}[1]{\markleft{\thesection\ #1}}
\ofoot{\pagemark}
\cfoot{}

%%%%%%%%%%%%

%%%%%%%%%%%% FINAL HEADER MATERIAL

% clereref must be loaded after anything that changes the referencing system
\usepackage{cleveref}
\creflabelformat{equation}{#2#1#3}

% make the code float work with cleverref
\crefname{codecell}{code}{codes}
\Crefname{codecell}{code}{codes}
% make the text float work with cleverref
\crefname{textcell}{text}{texts}
\Crefname{textcell}{text}{texts}
% make the text float work with cleverref
\crefname{errorcell}{error}{errors}
\Crefname{errorcell}{error}{errors}

%%%%%%%%%%%%

\begin{document}

    \begin{titlepage}
  \begin{flushright}
    \includegraphics[width=0.7\textwidth]{Uganda_logit_model_files/510logo.png}
  \end{flushright}

  \begin{center}

  \vspace*{1cm}

  \Huge\textbf{Predicting Droughts in Uganda}

  \vspace{0.5cm}\LARGE{from meteorological satalite data}

  \vspace{1.5cm}

  \begin{minipage}{0.8\textwidth}
    \begin{center}
    \begin{minipage}{0.39\textwidth}
    \begin{flushleft} \Large
    \emph{\GetTranslation{Author}:}\\Behrouz Eslami and Misha Klein\\\href{mailto:behrooz.eslami@gmail.com}{behrooz.eslami@gmail.com}
    \end{flushleft}
    \end{minipage}
    \hspace{\fill}
    \begin{minipage}{0.39\textwidth}
    \begin{flushright} \Large
    \end{flushright}
    \end{minipage}
    \end{center}
  \end{minipage}

  \vfill

  \begin{minipage}{0.8\textwidth}
  \begin{center}
  \end{center}
  \end{minipage}

  \vspace{0.8cm}

  \vspace{0.4cm}

  \today

  \end{center}
  \end{titlepage}

    \begingroup
    \let\cleardoublepage\relax
    \let\clearpage\relax
    \endgroup

\hypertarget{summary}{%
\section{Summary}\label{summary}}

Here we train and optimize a logistic regression model to predict the
occurrence of droughts in Uganda based of meteorological data. The
precipitation and the \textbf{E}nhanced \textbf{V}egitation
\textbf{I}ndex (EVI) turn out to be the most important drought
predictors among all the meteorological quantities that we studied,
implying that the drought reports are triggered by both the hydrological
and agricultural factors. The model is able to capture 70 percent of the
reported droughts in our historical data. Based on our model, we derive
a ``drought score'' which would serve as a potentially useful metric to
monitor and predict droughts in Uganda. We show that this drought score
also strongly correlates with the Dry Matter Productivity (DMP) data.

\hypertarget{data-collection}{%
\section{Data collection}\label{data-collection}}

Our aim is to construct a model to predict the occurrence of droughts in
Uganda, using meteorological satellite data. The meteorological
parameters that serve as drought indicators, and to which we refer as
\textbf{features}, are collected using the
\href{https://developers.google.com/earth-engine/python_install}{Google
Earth engine} by this
\href{https://github.com/rodekruis/IBF-system/blob/master/trigger-model-development/drought/skill-assessment/GoogleEarthEngine/GEE_get_data.py}{python
script}, and are listed bellow

\begin{itemize}
\tightlist
\item
  \href{https://developers.google.com/earth-engine/datasets/catalog/MODIS_006_MOD13A2}{\textbf{NDVI:}}
  \emph{`\textbf{\emph{N}}ormalized \textbf{\emph{D}}ifference
  \textbf{\emph{V}}egitation \textbf{\emph{I}}ndex'} (scale by
  0.0001).\\
\item
  \href{https://developers.google.com/earth-engine/datasets/catalog/MODIS_006_MOD13A2}{\textbf{EVI:}}
  \emph{`\textbf{\emph{E}}nhanced \textbf{\emph{V}}egitation
  \textbf{\emph{I}}ndex'} (scale by 0.0001).
\item
  \href{https://developers.google.com/earth-engine/datasets/catalog/TRMM_3B43V7\#bands}{\textbf{precipitation:}}:
  Rainfall in mm/hrs. Name in our dataset:
  `\texttt{precipitation\ per\ hour\ v1}'.
\item
  \href{https://developers.google.com/earth-engine/datasets/catalog/JAXA_GPM_L3_GSMaP_v6_operational}{\textbf{hourlyPrecipRate:}}
  Rainfall measured in mm/hrs. Name in our dataset:
  `\texttt{precipitation\ per\ hour\ v2}'.
\item
  \href{https://developers.google.com/earth-engine/datasets/catalog/MODIS_006_MOD11A1}{\textbf{LST\_Day\_1km:}}
  Land surface temperature during daytime in 50 Kelvins. Name in our
  dataset: `\texttt{surface\ temperature\ daytime}'.
\item
  \href{https://developers.google.com/earth-engine/datasets/catalog/MODIS_006_MOD11A1}{\textbf{LST\_Night\_1km:}}
  Land surface temperature during daytime in 50 Kelvins. Name in our
  dataset: `\texttt{surface\ temperature\ nighttime}'.
\item
  \href{https://developers.google.com/earth-engine/datasets/catalog/NASA_FLDAS_NOAH01_C_GL_M_V001}{\textbf{Evap\_tavg:}}
  Evapotranspiration measured in kg m\^{}\{-2\} s\^{}\{-1\}. Name in our
  dataset: `\texttt{evapotranspiration}'.
\item
  \href{https://developers.google.com/earth-engine/datasets/catalog/NASA_FLDAS_NOAH01_C_GL_M_V001}{\textbf{Rainf\_f\_tavg:}}
  Precipitation rate measured in kg m\^{}\{-2\} s\^{}\{-1\} Name in our
  dataset: `\texttt{rainfall}'.
\item
  \href{https://developers.google.com/earth-engine/datasets/catalog/NASA_FLDAS_NOAH01_C_GL_M_V001}{\textbf{SoilMoi00\_10cm\_tavg:}}
  Soil moisture (0 - 10 cm underground) in m\^{}3 m-3. Name in our
  dataset: `\texttt{SoilMoisture00\ 10cm}'.
\item
  \href{https://developers.google.com/earth-engine/datasets/catalog/NASA_FLDAS_NOAH01_C_GL_M_V001}{\textbf{SoilMoi10\_40cm\_tavg:}}
  Soil moisture (10 - 40 cm underground) in m\^{}3 m-3. Name in our
  dataset: `\texttt{SoilMoisture10\ 40cm}'.
\item
  \href{https://developers.google.com/earth-engine/datasets/catalog/NASA_FLDAS_NOAH01_C_GL_M_V001}{\textbf{SoilMoi40\_100cm\_tavg:}}
  Soil moisture (40 - 100 cm underground) in m\^{}3 m-3. Name in our
  dataset: `\texttt{SoilMoisture40\ 100cm}'.
\item
  \href{https://developers.google.com/earth-engine/datasets/catalog/NASA_FLDAS_NOAH01_C_GL_M_V001}{\textbf{SoilMoi100\_200cm\_tavg:}}
  Soil moisture (100 - 200 cm underground) in m\^{}3 m-3. Name in our
  dataset: `\texttt{SoilMoisture100\ 200cm}'.
\item
  \href{https://developers.google.com/earth-engine/datasets/catalog/NASA_FLDAS_NOAH01_C_GL_M_V001}{\textbf{SoilTemp00\_10cm\_tavg:}}
  Soil temperature (0 - 10 cm underground) in K. Name in our dataset:
  `\texttt{SoilTemperature00\ 10cm}'.
\item
  \href{https://developers.google.com/earth-engine/datasets/catalog/NASA_FLDAS_NOAH01_C_GL_M_V001}{\textbf{SoilTemp10\_40cm\_tavg:}}
  Soil temperature (10 - 40 cm underground) in K. Name in our dataset:
  `\texttt{SoilTemperature10\ 40cm}'.
\item
  \href{https://developers.google.com/earth-engine/datasets/catalog/NASA_FLDAS_NOAH01_C_GL_M_V001}{\textbf{SoilTemp40\_100cm\_tavg:}}
  Soil temperature (40 - 100 cm underground) in K. Name in our dataset:
  `\texttt{SoilTemperature40\ 100cm}'.
\item
  \href{https://developers.google.com/earth-engine/datasets/catalog/NASA_FLDAS_NOAH01_C_GL_M_V001}{\textbf{SoilTemp100\_200cm\_tavg:}}
  Soil temperature (100 - 200 cm underground) in K. Name in our dataset:
  `\texttt{SoilTemperature100\ 200cm}'.
\item
  \href{https://developers.google.com/earth-engine/datasets/catalog/NASA_FLDAS_NOAH01_C_GL_M_V001}{\textbf{Tair\_f\_tavg:}}
  Near surface air temperature in K. Name in our dataset:
  `\texttt{air\ temperature}'.
\item
  \href{https://developers.google.com/earth-engine/datasets/catalog/NASA_FLDAS_NOAH01_C_GL_M_V001}{\textbf{Wind\_f\_tavg:}}
  Near surface wind speed in m/s. Name in our dataset:
  `\texttt{wind\ speed}'.\\
\item
  \href{https://spei.csic.es/}{\textbf{SPEI} columns:} We calculated the
  \emph{`\textbf{\emph{S}}tandardized
  \textbf{\emph{P}}recipitation-\textbf{\emph{E}}vapotranspiration
  \textbf{\emph{I}}ndex'} from the columns `\texttt{rainfall}' and
  `\texttt{evapotranspiration}' in our dataset, using the published
  \href{https://spei.csic.es/tools.html}{R package}. We have used
  different \emph{scale} parameters from 1 to 12 months, generating the
  columns `\texttt{SPEI\ 1month}' to `\texttt{SPEI\ 12month}'. The R
  script to perform this calculation can be find
  \href{https://github.com/rodekruis/IBF-system/blob/master/trigger-model-development/drought/skill-assessment/Model\%20Event\%20Data/SPEI_satellite/calc_SPEI.R}{here}.
\end{itemize}

All features were averaged over each district (i.e.~admin. level 1), and
either averaged or summed in successive one-month intervals (see this
\href{https://github.com/rodekruis/IBF-system/blob/master/trigger-model-development/drought/skill-assessment/GoogleEarthEngine/GEE_get_data.py}{python
script} for details).

The target variable that we are trying to predict, to which we refer to
as the \textbf{label}, is the occurrence of droughts at a certain time
point/period in a particular district. This would be a binary
(i.e.~Boolean) variable, taking the value `True' when a drought has
occurred, and `False' when no drought is reported. This information
comes from two different sources, the Red Cross desinventar and news
articles, both can be found
\href{https://rodekruis.sharepoint.com/sites/510-CRAVK-510/Gedeelde\%20\%20documenten/Forms/AllItems.aspx?RootFolder=\%2Fsites\%2F510\%2DCRAVK\%2D510\%2FGedeelde\%20\%20documenten\%2F\%5BCTRY\%5D\%20Uganda\%2F2\%2D\%20IBF\%20DROUGHT\%20UGANDA\%2FDROUGHT\%20EVENTS\%20\%26\%20IMPACT\%20ANALYSIS\&FolderCTID=0x012000D122CA8FA5B68345A13A32EDAA653915}{here}
(see Figure \ref{fig_drought_reports}). We chose to build the label
variable by combining the two sources together. This is represented by
the column `\texttt{drought\ reported}' in the dataset.

We merged all the features and labels into a single dataset using this
\href{https://github.com/rodekruis/IBF-system/blob/master/trigger-model-development/drought/skill-assessment/Model\%20Event\%20Data/combining\%20satellite\%20data\%20with\%20desinventar\%20events\%20and\%20news\%20article\%20events\%20.ipynb}{Jupyter
notebook}. The resultant dataset can be found
\href{https://github.com/rodekruis/IBF-system/blob/master/trigger-model-development/drought/skill-assessment/Model\%20Event\%20Data/datasets/Droughts_satelite_and_events.csv}{here}.
Each row of the dataset corresponds to a unique pair of date and
district.

\begin{figure}[H]\begin{center}\adjustimage{max size={0.9\linewidth}{0.9\paperheight},height=0.25\paperheight}{Uganda_logit_model_files/output_4_0.png}\end{center}\caption{Distribution of the drought reports (from Red Cross desinventar and news
articles) between 2000 and 2019 across Uganda districts.}\label{fig_drought_reports}\end{figure}

\hypertarget{data-preprocessing}{%
\section{Data preprocessing}\label{data-preprocessing}}

We consider the bimodal
\href{https://fews.net/east-africa/uganda}{agricultural calendar for
Uganda} (Figure \ref{fig:calendar}, top panel) which has two harvest
seasons per year: the first one during June and July, and the second one
in November and December.

\begin{figure}[]
\hypertarget{fig:calendar}{%
\begin{center}
\adjustimage{max size={0.9\linewidth}{0.9\paperheight},width=1.0\linewidth}{./pictures/seasonal-calendar-uganda.png}
\end{center}
\caption{agricultural calendar for Uganda}\label{fig:calendar}
}
\end{figure}

We set as our goal to predict whether or not a drought occurs within
each harvest season, based on the meteorological data in the three-month
period preceding the season. To this end we redefine the label variable
to have the value \texttt{True} if a drought is reported in any of the
two months of the season, and \texttt{False} otherwise. We also average
all the meteorological data, except SPEI, over the preceding three-month
period and use them as features for our model. For SPEI we use the
column `\texttt{SPEI\ 3month}' and take the data points that correspond
to the three-month period before each harvest season. We then normalize
all features except \texttt{SPEI} per district and season. This means
that for any given district and season, we calculate the mean and the
standard deviation of each feature during all years, and then
standardize the feature by subtracting the mean and dividing by the
standard deviation. The \texttt{SPEI} feature is already normalized by
construction.\\
The transformed dataset can be downloaded from
\href{https://github.com/rodekruis/IBF-system/blob/master/trigger-model-development/drought/skill-assessment/Model\%20Event\%20Data/datasets/Uganda_seasonal_normalized.csv}{here}.
Each row is indicated by a unique triple of district, year and season.
The first harvest season is indicated by \texttt{6\_7} and the second
one by \texttt{11\_12}.

As a final step of preparing the training data, we decided to only keep
those rows in the dataset that correspond to either a year with a
reported drought or a year preceding/following a drought. This will
potentially reduce the noise in the data due to the droughts that have
actually happened but remained unreported.

\hypertarget{data-exploration}{%
\section{Data exploration}\label{data-exploration}}

\hypertarget{correlations-among-the-features}{%
\subsection{Correlations among the
features}\label{correlations-among-the-features}}

Many of the features in our dataset are strongly correlated. To
investigate this, we used a
\href{https://scikit-learn.org/stable/modules/biclustering.html}{biclustering
algorithm} to divide the features into three separate groups. Features
within a group correlate strongly with each other, while the correlation
between features from different groups is weaker. Figure \ref{corr_mat}
Shows the resultant correlation matrix, where the colors represent the
absolute value of the correlation coefficient. We can see that the soil
moisture and soil temperature features all correlate strongly with each
other, as expected. Vegetation indices EVI and NDVI correlate well with
each other, with air and surface temperatures, and with the soil
temperature (but not with the soil moisture). Surprisingly the features
related to precipitation, namely `\texttt{rainfall}',
`\texttt{precipitation\_per\_hour\_v1}' and
`\texttt{precipitation\_per\_hour\_v2}', do not correlate well with each
other. The reason for this is not clear for us.

\begin{figure}[H]\begin{center}\adjustimage{max size={0.9\linewidth}{0.9\paperheight},height=0.4\paperheight}{Uganda_logit_model_files/output_16_0.png}\end{center}\caption{Correlations matrix for all pairs of features. The colors represent the
absolute value of the correlation coefficient.}\label{corr_mat}\end{figure}

The correlation plots among the features within each of the three groups
are shown in Figures \ref{pairplot1} to \ref{pairplot3}.

\begin{figure}[H]\begin{center}\adjustimage{max size={0.9\linewidth}{0.9\paperheight},height=0.55\paperheight}{Uganda_logit_model_files/output_18_0.png}\end{center}\caption{Correlation plots for the features in group 1. The points are colored
based on their label values.}\label{pairplot1}\end{figure}

\begin{figure}[H]\begin{center}\adjustimage{max size={0.9\linewidth}{0.9\paperheight},height=0.55\paperheight}{Uganda_logit_model_files/output_19_0.png}\end{center}\caption{Correlation plots for the features in group 2. The points are colored
based on their label values.}\label{pairplot2}\end{figure}

\begin{figure}[H]\begin{center}\adjustimage{max size={0.9\linewidth}{0.9\paperheight},height=0.55\paperheight}{Uganda_logit_model_files/output_20_0.png}\end{center}\caption{Correlation plots for the features in group 3. The points are colored
based on their label values.}\label{pairplot3}\end{figure}

\hypertarget{correlation-with-the-target}{%
\subsection{Correlation with the
target}\label{correlation-with-the-target}}

Figures \ref{fig_dist1} to \ref{fig_dist19} show the distribution of
each feature separately for `\texttt{drought\ reported\ =False}' (blue)
and `\texttt{drought\ reported\ =\ True}' (orange). The medians of the
distributions, as well as the absolute values of the difference between
them, are reported in Table \ref{tlabe_median}. The features are ranked
in descending order according to the difference in medians. From Table
\ref{tlabe_median} it is clear that the features that are related to
precipitation, either directly or indirectly, are important drought
predictors.

\begin{table}[H]
\caption{The median values of the features grouped by the label. The column
`difference' shows the absolute values of the difference between the two
groups.}\label{tlabe_median}
\centering
\begin{adjustbox}{max width=\textwidth}\rowcolors{2}{gray!20}{white}
\begin{tabular}{llrrr}
\toprule
{} &                        feature &  no drought &  yes drought &  difference \\
\midrule
0  &            SoilMoisture00 10cm &    0.280299 &    -0.570226 &    0.850525 \\
1  &      precipitation per hour v2 &    0.063213 &    -0.698218 &    0.761430 \\
2  &            SoilMoisture10 40cm &    0.216020 &    -0.530722 &    0.746742 \\
3  &           SoilMoisture40 100cm &    0.203412 &    -0.531513 &    0.734925 \\
4  &             evapotranspiration &    0.044880 &    -0.602795 &    0.647675 \\
5  &                       rainfall &    0.158049 &    -0.486975 &    0.645024 \\
6  &       SoilTemperature100 200cm &   -0.268359 &     0.320006 &    0.588365 \\
7  &        SoilTemperature40 100cm &   -0.224721 &     0.361192 &    0.585913 \\
8  &         SoilTemperature10 40cm &   -0.211171 &     0.302455 &    0.513625 \\
9  &         SoilTemperature00 10cm &   -0.144578 &     0.340174 &    0.484752 \\
10 &                            EVI &    0.214773 &    -0.196214 &    0.410986 \\
11 &                    SPEI 3month &    0.211545 &    -0.192696 &    0.404242 \\
12 &                     wind speed &   -0.024963 &     0.329311 &    0.354275 \\
13 &      precipitation per hour v1 &   -0.240641 &    -0.549282 &    0.308641 \\
14 &                           NDVI &    0.141345 &    -0.127157 &    0.268502 \\
15 &          SoilMoisture100 200cm &   -0.036950 &    -0.292951 &    0.256001 \\
16 &    surface temperature daytime &   -0.092329 &     0.155518 &    0.247847 \\
17 &                air temperature &   -0.153588 &     0.028358 &    0.181946 \\
18 &  surface temperature nighttime &    0.090008 &     0.235946 &    0.145938 \\
\bottomrule
\end{tabular}

\end{adjustbox}
\end{table}

\begin{figure}[H]\begin{center}\adjustimage{max size={0.9\linewidth}{0.9\paperheight},height=0.17\paperheight}{Uganda_logit_model_files/output_24_0.png}\end{center}\caption{Distribution of `\texttt{SoilMoisture00\ 10cm}' separated by the label
values.}\label{fig_dist1}\end{figure}

\begin{figure}[H]\begin{center}\adjustimage{max size={0.9\linewidth}{0.9\paperheight},height=0.17\paperheight}{Uganda_logit_model_files/output_25_0.png}\end{center}\caption{Distribution of `\texttt{precipitation\ per\ hour\ v2}' separated by the
label values.}\label{fig_dist2}\end{figure}

\begin{figure}[H]\begin{center}\adjustimage{max size={0.9\linewidth}{0.9\paperheight},height=0.17\paperheight}{Uganda_logit_model_files/output_26_0.png}\end{center}\caption{Distribution of `\texttt{SoilMoisture10\ 40cm}' separated by the label
values.}\label{fig_dist3}\end{figure}

\begin{figure}[H]\begin{center}\adjustimage{max size={0.9\linewidth}{0.9\paperheight},height=0.17\paperheight}{Uganda_logit_model_files/output_27_0.png}\end{center}\caption{Distribution of `\texttt{SoilMoisture40\ 100cm}' separated by the label
values.}\label{fig_dist4}\end{figure}

\begin{figure}[H]\begin{center}\adjustimage{max size={0.9\linewidth}{0.9\paperheight},height=0.17\paperheight}{Uganda_logit_model_files/output_28_0.png}\end{center}\caption{Distribution of `\texttt{evapotranspiration}' separated by the label
values.}\label{fig_dist5}\end{figure}

\begin{figure}[H]\begin{center}\adjustimage{max size={0.9\linewidth}{0.9\paperheight},height=0.17\paperheight}{Uganda_logit_model_files/output_29_0.png}\end{center}\caption{Distribution of `\texttt{rainfall}' separated by the label values.}\label{fig_dist6}\end{figure}

\begin{figure}[H]\begin{center}\adjustimage{max size={0.9\linewidth}{0.9\paperheight},height=0.17\paperheight}{Uganda_logit_model_files/output_30_0.png}\end{center}\caption{Distribution of `\texttt{SoilTemperature100\ 200cm}' separated by the
label values.}\label{fig_dist7}\end{figure}

\begin{figure}[H]\begin{center}\adjustimage{max size={0.9\linewidth}{0.9\paperheight},height=0.17\paperheight}{Uganda_logit_model_files/output_31_0.png}\end{center}\caption{Distribution of `\texttt{SoilTemperature40\ 100cm}' separated by the
label values.}\label{fig_dist8}\end{figure}

\begin{figure}[H]\begin{center}\adjustimage{max size={0.9\linewidth}{0.9\paperheight},height=0.17\paperheight}{Uganda_logit_model_files/output_32_0.png}\end{center}\caption{Distribution of `\texttt{SoilTemperature10\ 40cm}' separated by the
label values.}\label{fig_dist9}\end{figure}

\begin{figure}[H]\begin{center}\adjustimage{max size={0.9\linewidth}{0.9\paperheight},height=0.17\paperheight}{Uganda_logit_model_files/output_33_0.png}\end{center}\caption{Distribution of `\texttt{SoilTemperature00\ 10cm}' separated by the
label values.}\label{fig_dist10}\end{figure}

\begin{figure}[H]\begin{center}\adjustimage{max size={0.9\linewidth}{0.9\paperheight},height=0.17\paperheight}{Uganda_logit_model_files/output_34_0.png}\end{center}\caption{Distribution of `\texttt{EVI}' separated by the label values.}\label{fig_dist11}\end{figure}

\begin{figure}[H]\begin{center}\adjustimage{max size={0.9\linewidth}{0.9\paperheight},height=0.17\paperheight}{Uganda_logit_model_files/output_35_0.png}\end{center}\caption{Distribution of `\texttt{SPEI\ 3month}' separated by the label values.}\label{fig_dist12}\end{figure}

\begin{figure}[H]\begin{center}\adjustimage{max size={0.9\linewidth}{0.9\paperheight},height=0.17\paperheight}{Uganda_logit_model_files/output_36_0.png}\end{center}\caption{Distribution of `\texttt{wind\ speed}' separated by the label values.}\label{fig_dist13}\end{figure}

\begin{figure}[H]\begin{center}\adjustimage{max size={0.9\linewidth}{0.9\paperheight},height=0.17\paperheight}{Uganda_logit_model_files/output_37_0.png}\end{center}\caption{Distribution of `\texttt{precipitation\ per\ hour\ v1}' separated by the
label values.}\label{fig_dist14}\end{figure}

\begin{figure}[H]\begin{center}\adjustimage{max size={0.9\linewidth}{0.9\paperheight},height=0.17\paperheight}{Uganda_logit_model_files/output_38_0.png}\end{center}\caption{Distribution of `\texttt{NDVI}' separated by the label values.}\label{fig_dist15}\end{figure}

\begin{figure}[H]\begin{center}\adjustimage{max size={0.9\linewidth}{0.9\paperheight},height=0.17\paperheight}{Uganda_logit_model_files/output_39_0.png}\end{center}\caption{Distribution of `\texttt{SoilMoisture100\ 200cm}' separated by the label
values.}\label{fig_dist16}\end{figure}

\begin{figure}[H]\begin{center}\adjustimage{max size={0.9\linewidth}{0.9\paperheight},height=0.17\paperheight}{Uganda_logit_model_files/output_40_0.png}\end{center}\caption{Distribution of `\texttt{surface\ temperature\ daytime}' separated by
the label values.}\label{fig_dist17}\end{figure}

\begin{figure}[H]\begin{center}\adjustimage{max size={0.9\linewidth}{0.9\paperheight},height=0.17\paperheight}{Uganda_logit_model_files/output_41_0.png}\end{center}\caption{Distribution of `\texttt{air\ temperature}' separated by the label
values.}\label{fig_dist18}\end{figure}

\begin{figure}[H]\begin{center}\adjustimage{max size={0.9\linewidth}{0.9\paperheight},height=0.17\paperheight}{Uganda_logit_model_files/output_42_0.png}\end{center}\caption{Distribution of `\texttt{surface\ temperature\ nighttime}' separated by
the label values.}\label{fig_dist19}\end{figure}

\hypertarget{fitting-and-optimizing-a-logistic-regression-model}{%
\section{Fitting and optimizing a Logistic Regression
model}\label{fitting-and-optimizing-a-logistic-regression-model}}

We train a
\href{https://scikit-learn.org/stable/modules/generated/sklearn.linear_model.LogisticRegression.html}{logistic
regression classifier} with
\href{https://scikit-learn.org/stable/auto_examples/linear_model/plot_logistic_l1_l2_sparsity.html}{\(l_1\)
penalty} to predict the occurrence of droughts in Uganda. Our goal is to
separate the rows of the dataset into two classes: a positive class
which represents the occurrence of a drought,
(\texttt{drought\ =\ True}), and a negative class where the model
predicts that no drought will occur (\texttt{drought\ =\ False}). It is
important to note that the population of the positive class
\(n^{pos}= 65\) in our training data is about 5 times smaller the
population of the negative class \(n^{neg}= 297\). The negative class is
therefore largely overrepresented, and a model trained on such data will
be biased towards predicting the negative class correctly, without
caring so much about capturing the positive class. Since having a good
predicting power for the positive class is equally important for us, we
need to correct for this imbalance in the class populations. We do this
by assigning unequal weights \(W^{pos}\) and \(W^{neg}\) to the positive
and the negative classes, which are inversely proportional to the class
populations:

\begin{equation}\label{eqn_wpos}
W^{pos}=\frac{\frac{1}{n^{pos}}}{\frac{1}{n^{pos}}+\frac{1}{n^{neg}}},
\end{equation}

and

\begin{equation}\label{eqn_wneg}
W^{neg}=\frac{\frac{1}{n^{neg}}}{\frac{1}{n^{pos}}+\frac{1}{n^{neg}}}.
\end{equation}

Feeding these weights into the logistic regression model guarantees that
both classes are represented equally when the model is trained on the
data.

The \(l_1\) penalty in the logistic regression model is associated with
a regularization parameter \(C\), which tunes the model complexity by
controlling the number of features that are included in the model. In
the framework of logistic regression, the contribution of each feature
is determined by a coefficient, and a zero coefficient means that the
corresponding feature does not enter the model. The regularization
parameter determines how many of these coefficients are different from
zero. Figure \ref{fig_l1} shows the model coefficients for all features
as a function of the regularization parameter \(C\). We can see that for
large values of \(C\) all coefficients are non-zero. As \(C\) decreases,
the coefficients gradually shrink to zero.

\begin{figure}[H]\begin{center}\adjustimage{max size={0.9\linewidth}{0.9\paperheight},height=0.27\paperheight}{Uganda_logit_model_files/output_51_0.png}\end{center}\caption{Model coefficients for all features as a function of the regularization
parameter \(C\).}\label{fig_l1}\end{figure}

Choosing an appropriate value for the regularization parameter is
important, as this systematically eliminates unimportant and redundant
features, so that only the most important features remain in the model.
We performed
\href{https://towardsdatascience.com/why-and-how-to-cross-validate-a-model-d6424b45261f}{3-fold
cross-validation} to find the best value for \(C\), where we used the
average \href{https://en.wikipedia.org/wiki/F1_score}{F1 score} as the
goodness of fit measure. The average F1 score is given by the weighted
sum of the F1 scores for the positive and negative classes:

\begin{equation}\label{eqn_F}
F1=W^{pos}F1^{pos}+W^{neg}F1^{neg},
\end{equation}

Where \(F1^{pos}\) and \(F1^{neg}\) are the F1 scores for the positive
and negative classes, and \(W^{pos}\) and \(W^{neg}\) are given by
equations \ref{eqn_wpos} and \ref{eqn_wneg} respectively.

To be sure that the outcome of the cross-validation is robust and
generalizable, we repeated the procedure 10 times where we randomly
shuffled the data before each round of cross-validation. The outcome is
demonstrated in Figure \ref{fig_cv}, where the curves in pail colors
show the cross-validation average F1 scores as a function of \(C\) for
each round of shuffling, and the mean of all these curves is shown in
dark black. The training average F1 score is also shown in the Figure
(dashed-dot curve, green).

\begin{figure}[H]\begin{center}\adjustimage{max size={0.9\linewidth}{0.9\paperheight},height=0.22\paperheight}{Uganda_logit_model_files/output_56_0.png}\end{center}\caption{Average F1 score as a function of the regularization parameter \(C\).
The curves in pail colors show the cross-validation average F1 scores
after randomly shuffling the data, and their mean is shown in dark
black. The dashed-dot curve (green) corresponds to the average F1 score
for the training data. The blue and red vertical dashed lines indicate
\(C=0.09\) and \(C=0.23\) respectively.}\label{fig_cv}\end{figure}

Looking at the dark black curve in Figure \ref{fig_cv}, it is clear that
for \(C<0.09\) (blue vertical dashed line) the cross-validation F1 score
rapidly grows by increasing \(C\). For values between 0.09 and 0.23 (red
vertical dashed line), the cross-validation average F1 score remains
relatively constant around 0.42, and the same happens for the training
average F1 score (see dashed-dot curve, green). For \(C>0.23\), although
the training score continues to grow and saturates at 0.57, the
cross-validation score always remain below the training scores and
quickly saturates at 0.48 after a mild increase. This implies that for
values of \(C>0.23\) the model stops to nicely generalize to new data.
In other words, the model would have the tendency to overfit to the
training data.

\begin{table}[H]
\caption{Nonzero model coefficients for \(C=0.3\). The model overfits to the
training data.}\label{tab_overfit}
\centering
\begin{adjustbox}{max width=\textwidth}\rowcolors{2}{gray!20}{white}
\begin{tabular}{lr}
\toprule
{} &  coefficients \\
feature                   &               \\
\midrule
precipitation per hour v2 &         -0.45 \\
EVI                       &         -0.27 \\
SoilMoisture100 200cm     &          0.07 \\
air temperature           &         -0.04 \\
evapotranspiration        &         -0.02 \\
\bottomrule
\end{tabular}

\end{adjustbox}
\end{table}

In the case of our drought model, detecting the overfitting is rather
easy, as it leads to intuitively nonsensical models. See, for example,
the model coefficients for \(C=0.3\) which are listed in Table
\ref{tab_overfit}. The coefficient for precipitation is negative, which
makes sense, as the precipitation should anticorrelate with droughts.
However, the positive coefficient for soil moisture and the negative
coefficient for air temperature do not make sense intuitively,
indicating that the model is fine-tuning itself to the training data. We
find that we get more and more of such nonsensical coefficients as we
increase \(C\) further. Since we inferred from Figure \ref{fig_cv} that
the model tends to overfit for \(C>0.23\), and the data in Table
\ref{tab_overfit} confirms that the overfitting already happens for
values as small as 0.3, In the rest of this analysis we use the optimal
value of \(C=0.23\) for the regularization parameter.

\hypertarget{model-evaluation}{%
\section{Model evaluation}\label{model-evaluation}}

A logistic regression model produces a score as a linear combination of
the features, where the contribution of each feature is determined by
its associated coefficient. Table \ref{tab_coefs} lists the nonzero
coefficients for our optimal drought model with \(C=0.23\).
Interestingly, only two features,
`\texttt{precipitation\ per\ hour\ v2}' and `\texttt{EVI}', enter the
optimal model, implying that the drought reports are triggered by both
the hydrological and agricultural factors. According to Table
\ref{tab_coefs}, the score produced by our optimal drought model can
therefore be written as

\begin{equation}\label{eqn_score}
\textrm{score}=-0.38\,\times\,\textrm{'precipitation per hour v2'}-0.23\,\times\,\textrm{'EVI'}.
\end{equation}

\begin{table}[H]
\caption{Nonzero model coefficients for the optimal model with \(C=0.23\).}\label{tab_coefs}
\centering
\begin{adjustbox}{max width=\textwidth}\rowcolors{2}{gray!20}{white}
\begin{tabular}{lr}
\toprule
{} &  coefficients \\
feature                   &               \\
\midrule
precipitation per hour v2 &         -0.38 \\
EVI                       &         -0.23 \\
\bottomrule
\end{tabular}

\end{adjustbox}
\end{table}

Note that, as mentioned earlier, in equation \ref{eqn_score} both
features are normalized per district and per season. The score can be
positive or negative, and is a measure for the likelihood of droughts.
The more positive the score is, the higher is the chance that a drought
occurs. Figure \ref{fig_score} shows the distribution of this ``drought
score'' separately for `\texttt{drought\ reported\ =False}' (blue) and
`\texttt{drought\ reported\ =\ True}' (orange), confirming that the
score correlates with the target.

\begin{figure}[H]\begin{center}\adjustimage{max size={0.9\linewidth}{0.9\paperheight},height=0.17\paperheight}{Uganda_logit_model_files/output_66_0.png}\end{center}\caption{Distribution of the drought score separated by the label values.}\label{fig_score}\end{figure}

The \href{https://en.wikipedia.org/wiki/Confusion_matrix}{confusion
matrix} and the accuracy associated with the optimal model, as well as
\href{https://en.wikipedia.org/wiki/Precision_and_recall}{precision,
recall} and the F1 scores for positive and negative classes, are given
in Text \ref{tab_confusion}. The model reaches the overall accuracy of
60\%, and has an impressive recall of \textasciitilde{} 70\% for the
positive class (which means that the model captures 70\% of the reported
droughts). On the other hand, the precision for the positive class is
rather low: about \%25. This is partially due to the overrepresentation
of the negative class, which leads to many false positives, and lowers
the precision for the positive class as well as the recall for the
negative class. We note however that there is a rather high chance that
an actual drought in Uganda remains unreported. We therefore expect that
the number of false positives presented in Text \ref{tab_confusion} is
an overestimation. The actual predictive power of the model can only be
assessed after testing it in the field.

\begin{textcell}[H]
\caption{Confusion matrix, accuracy, and the goodness of fit metrics for the
optimal model.}
\label{tab_confusion}
\begin{lstlisting}[aboveskip=5pt,basicstyle=\small,belowskip=5pt,breakindent=0pt,language={},numbers=none,postbreak={},xrightmargin=7pt]
                 Confusion matrix
                 Score positive    Score negative
Actual positive        45                20
Actual negative       127               170

Accuracy  0.59

           Positive      Negative
Num case       65           297
Precision    0.26          0.89
Recall       0.69          0.57
F-score      0.38          0.70

Weighted Average F-score  0.44

\end{lstlisting}\end{textcell}

To have a baseline with which we can compare the performance of our
model, we consider a ``random model'' which classifies the data by
``coin flipping'': irrespective to the features values, the model
assigns \texttt{True} and \texttt{False} labels to the data points with
equal probability. Text \ref{tab_random} lists the fit quality metrics
for such a random model (averaged over 100 runs). We see that our model
clearly outperforms the random model in every aspect, especially in the
case of the positive class.

\begin{textcell}[H]
\caption{Goodness of fit metrics for a random model.}
\label{tab_random}
\begin{lstlisting}[aboveskip=5pt,basicstyle=\small,belowskip=5pt,breakindent=0pt,language={},numbers=none,postbreak={},xrightmargin=7pt]
           Positive      Negative
Precision    0.18          0.82
Recall       0.50          0.50
F_score      0.26          0.62

\end{lstlisting}\end{textcell}

Figures \ref{fig_pr} and \ref{fig_roc} show the
\href{https://acutecaretesting.org/en/articles/precision-recall-curves-what-are-they-and-how-are-they-used}{precision-recall
curve} and the
\href{https://en.wikipedia.org/wiki/Receiver_operating_characteristic}{receiver
operating characteristic (ROC)} for the optimal model. The area under
the ROC curve
(\href{https://towardsdatascience.com/understanding-auc-roc-curve-68b2303cc9c5}{auc})
is a metric commonly used to assess the predictive power of binary
classifiers. A perfect classifier has an auc of one, while for a random
classifier auc equals 0.5. Our model reaches auc of 0.7.

\begin{figure}[H]\begin{center}\adjustimage{max size={0.9\linewidth}{0.9\paperheight},height=0.2\paperheight}{Uganda_logit_model_files/output_72_0.png}\end{center}\caption{Precision-recall curve for the optimal model.}\label{fig_pr}\end{figure}

\begin{figure}[H]\begin{center}\adjustimage{max size={0.9\linewidth}{0.9\paperheight},height=0.2\paperheight}{Uganda_logit_model_files/output_73_0.png}\end{center}\caption{receiver operating characteristic for the optimal model.}\label{fig_roc}\end{figure}

\hypertarget{drought-prediction-successes-and-failures}{%
\section{Drought prediction: successes and
failures}\label{drought-prediction-successes-and-failures}}

\begin{figure}[H]\begin{center}\adjustimage{max size={0.9\linewidth}{0.9\paperheight},height=0.2\paperheight}{Uganda_logit_model_files/output_75_0.png}\end{center}\caption{Number of reported droughts in Uganda during each harvest season between
2000 and 2019. Major droughts are highlighted in blue (Related to
Figures \ref{fig_map2009} to \ref{fig_map2013b}).}\label{fig_droughtcount_seasonal}\end{figure}

Figure \ref{fig_droughtcount_seasonal} shows the number of reported
droughts in Uganda during each harvest season between 2000 and 2019.
Specifically, a large number of reports have been appeared in June-July
2009. Figure \ref{fig_map2009} shows the distribution of our drought
score (equation \ref{eqn_score}) over the whole contry during this
period: the score is generally positive and large. Conversely, in the
same harvest season at 2010 no drought has been reported, and Figure
\ref{fig_map2010} shows that the score is clearly lower. The model
similarly captures the reported droughts (highlighted in Figure
\ref{fig_droughtcount_seasonal}) in 2000 (Figure \ref{fig_map2000}),
2011 (Figure \ref{fig_map2011}) and 2016 (Figure \ref{fig_map2016}).

\begin{figure}[H]\begin{center}\adjustimage{max size={0.9\linewidth}{0.9\paperheight},height=0.3\paperheight}{Uganda_logit_model_files/output_77_0.png}\end{center}\caption{Model scores for Uganda districts in the first harvest season in 2009.
The colorbar is capped at +/- 0.6. Stars mark districts in which a
drought has been reported. The model successfully captures the droughts.}\label{fig_map2009}\end{figure}

\begin{figure}[H]\begin{center}\adjustimage{max size={0.9\linewidth}{0.9\paperheight},height=0.3\paperheight}{Uganda_logit_model_files/output_78_0.png}\end{center}\caption{Model scores for Uganda districts in the first harvest season in 2010.
The colorbar is capped at +/- 0.6. No drought has been reported, and the
model scores are generally lower compared to the same period in 2009
(see Figure \ref{fig_map2009}).}\label{fig_map2010}\end{figure}

\begin{figure}[H]\begin{center}\adjustimage{max size={0.9\linewidth}{0.9\paperheight},height=0.3\paperheight}{Uganda_logit_model_files/output_79_0.png}\end{center}\caption{Model scores for Uganda districts in the first harvest season in 2000.
The colorbar is capped at +/- 0.6. Stars mark districts in which a
drought has been reported. The model successfully captures the droughts.}\label{fig_map2000}\end{figure}

\begin{figure}[H]\begin{center}\adjustimage{max size={0.9\linewidth}{0.9\paperheight},height=0.3\paperheight}{Uganda_logit_model_files/output_80_0.png}\end{center}\caption{Model scores for Uganda districts in the first harvest season in 2011.
The colorbar is capped at +/- 0.6. Stars mark districts in which a
drought has been reported. The model successfully captures the droughts.}\label{fig_map2011}\end{figure}

\begin{figure}[H]\begin{center}\adjustimage{max size={0.9\linewidth}{0.9\paperheight},height=0.3\paperheight}{Uganda_logit_model_files/output_81_0.png}\end{center}\caption{Model scores for Uganda districts in the second harvest season in 2016.
The colorbar is capped at +/- 0.6. Stars mark districts in which a
drought has been reported. The model successfully captures the droughts.}\label{fig_map2016}\end{figure}

One notable exception is the period of June-July 2013 where the model
fails drastically (see Figure \ref{fig_map2013a}): it produces a low
score anywhere within the country while seven droughts has bin reported.
In an attempt to rationalize this, we slightly modified the drought
score: instead of considering the three-month period preceding June
2013, we calculated the score over the period of May to June 2013. As
Figure \ref{fig_map2013b} shows, this modification improves the model
performance in this particular instance. We therefore suggest that the
failure of the model might be due to an unusually rapid change in the
meteorological factors.

\begin{figure}[H]\begin{center}\adjustimage{max size={0.9\linewidth}{0.9\paperheight},height=0.3\paperheight}{Uganda_logit_model_files/output_83_0.png}\end{center}\caption{Model scores for Uganda districts in the first harvest season in 2013.
The colorbar is capped at +/- 0.6. Stars mark districts in which a
drought has been reported. The model fails to capture the droughts.}\label{fig_map2013a}\end{figure}

\begin{figure}[H]\begin{center}\adjustimage{max size={0.9\linewidth}{0.9\paperheight},height=0.3\paperheight}{Uganda_logit_model_files/output_85_0.png}\end{center}\caption{The same as \ref{fig_map2013a}, but the score is calculated by averaging
the features over the three-month period of May-July 20013. The model
performance is improved.}\label{fig_map2013b}\end{figure}

\hypertarget{generalizing-the-model-towards-a-drought-monitoring-tool}{%
\section{Generalizing the model towards a drought-monitoring
tool}\label{generalizing-the-model-towards-a-drought-monitoring-tool}}

We looked into the potential of our model as a drought-monitoring tool.
To this end, we generalize our drought score in such a way that it can
be calculated continuously over time, instead of being limited to just a
three-month period prior to any harvest season. This is done by
averaging the features using a sliding temporal window with a period of
three months, normalizing them per month and district as described in
Section \ref{data-collection}, and calculating the drought score using
equation \ref{eqn_score}.

\begin{figure}[H]\begin{center}\adjustimage{max size={0.9\linewidth}{0.9\paperheight},height=0.2\paperheight}{Uganda_logit_model_files/output_89_0.png}\end{center}\caption{Number of droughts in Uganda during each month between 2000 and 2019.
Orange curve: number of reported droughts. Blue cure: predicted number
of droughts by thresholding the draught score at zero. Some instances of
successful performance of the model are highlighted in blue.}\label{fig_droughtcount}\end{figure}

Figure \ref{fig_droughtcount} shows the total number of reported
droughts in Uganda (orange curve), for every month between 2000 and
2019, compared with our model prediction (blue curve) obtained by
thresholding the score at zero (i.e.~positive score: drought occurs;
negative score: drought does not occur). As expected from the large
number of false positives in Text \ref{tab_confusion}, our model
generally overestimates the number of droughts. Notably, however, there
are distinct time points where the model predicts a ``global drought''
in Uganda, where 100 or more districts out of the 120 districts of
Uganda (i.e.~more than 80\%) simultaneously have positive drought
scores. Interestingly, in many cases these predicted global droughts
coincide with a large number of reported droughts (several examples are
highlighted in blue in Figure \ref{fig_droughtcount}). We therefore
suggest that these timepoints are of particular importance where our
drought score, if implemented in a drought monitoring tool, can serve as
an warning for extensive droughts in Uganda. A crude example of such a
monitoring system is implemented in the end of this
\href{https://github.com/rodekruis/IBF-system/blob/master/trigger-model-development/drought/skill-assessment/Model\%20Event\%20Data/Uganda_logit_model.ipynb}{Jupyter
notebook} (see Figures \ref{fig_monitor} and \ref{fig_monitor_2}), where
users can explore the distribution of the drought score at any date they
chose, along with any reported drought if exists.

\begin{figure}[H]
\hypertarget{fig_monitor}{%
\begin{center}
\adjustimage{max size={0.9\linewidth}{0.9\paperheight},width=0.6\linewidth}{./pictures/drought_monitor_date.png}
\end{center}
\caption{A crude example of a simple drought monitor with a date picker.}\label{fig_monitor}
}
\end{figure}

\begin{figure}[H]
\hypertarget{fig_monitor_2}{%
\begin{center}
\adjustimage{max size={0.9\linewidth}{0.9\paperheight},width=0.6\linewidth}{./pictures/drought_monitor_slider.png}
\end{center}
\caption{Another example of a drought monitor with a date slider.}\label{fig_monitor_2}
}
\end{figure}

\hypertarget{comparison-of-the-drought-score-with-dry-matter-productivity}{%
\section{Comparison of the drought score with dry matter
productivity}\label{comparison-of-the-drought-score-with-dry-matter-productivity}}

Finally, we compared our model score with the Dry Matter Productivity
(DMP) \href{https://land.copernicus.eu/global/products/dmp}{data} for
Uganda at the district level and in one-month intervals (data extraction
performed in this
\href{https://github.com/rodekruis/IBF-system/blob/master/trigger-model-development/drought/skill-assessment/Model\%20Event\%20Data/Extracting_DMP_data.ipynb}{Jupyter
notebook} and the resultant dataset can be accessed
\href{https://github.com/rodekruis/IBF-system/blob/master/trigger-model-development/drought/skill-assessment/Model\%20Event\%20Data/datasets/Uganda_DMP_data.csv}{here}).
DMP represents the overall growth rate or dry biomass increase of the
vegetation, and should serve as a drought indicator. For convenience, we
define \(\Delta\)DMP as the difference between each value of DMP and the
largest measured value in the dataset:

\begin{equation}\label{eqn_dmp}
\Delta \textrm{DMP}\equiv \textrm{Maximum (DMP)}-\textrm{DMP}
\end{equation}

\begin{figure}[H]\begin{center}\adjustimage{max size={0.9\linewidth}{0.9\paperheight},height=0.22\paperheight}{Uganda_logit_model_files/output_98_0.png}\end{center}\caption{Correlation of the normalized \(\Delta\)DMP with the drought score
(left), precipitation (middle) and EVI (right).}\label{fig_dmp_corr}\end{figure}

We expect \(\Delta\)DMP to positively correlate with our drought score.
To confirm this, we applied the same preprocessing steps as outlined in
Section \ref{generalizing-the-model-towards-a-drought-monitoring-tool}
to \(\Delta\)DMP. That is, we averaged the data using a three-month
temporal window and then standardized the data per month and district.
We found a correlation of about \%60 between this ``normalized
\(\Delta\)DMP'' and our drought score (Figure \ref{fig_dmp_corr} left).
Both components of the sore, namely precipitation (Figure
\ref{fig_dmp_corr} middle) and EVI (Figure \ref{fig_dmp_corr} right)
contribute to the correlation. As an example, Figure \ref{fig_dmp_2017}
compares the drought score and normalized \(\Delta\)DMP across all
Uganda districts in January 2017, and Figure \ref{fig_dmp_abim} shows
the temporal variations of the two for the district of Abim. A good
correlation is evident in both Figures. In this
\href{https://github.com/rodekruis/IBF-system/blob/master/trigger-model-development/drought/skill-assessment/Model\%20Event\%20Data/Uganda_logit_model.ipynb}{Jupyter
notebook}, it is possible to interactively change these two
visualizations to explore different dates/districts.

We also found that the degree of correlation between the drought score
and the normalized \(\Delta\)DMP depends on location: Figure
\ref{fig_dmp_corr_districts} shows the correlation coefficient
separately for each district, and the distribution of all the
coefficients is displayed in Figure \ref{fig_dmp_corr_distribution}.
While for 70\% of the districts the correlation coefficient is higher
than 60\%, there are areas of adjacent districts with low correlation
coefficient. Interestingly, some of the districts with the lowest
correlation happen to be islands in Lake Victoria (compare Figure
\ref{fig_dmp_corr_districts} with Figure \ref{fig_ugandamap}).
Therefore, it is perhaps not surprising that the correlation breaks down
in those districts.

We hope that further investigation of the link between the drought score
and DMP would lead to an improvement of the predictive power of our
model.

\begin{figure}[H]\begin{center}\adjustimage{max size={0.9\linewidth}{0.9\paperheight},height=0.25\paperheight}{Uganda_logit_model_files/output_102_0.png}\end{center}\caption{Distribution of the drought score (left) and normalized \(\Delta\)DMP
(right) across Uganda districts in January 2017.}\label{fig_dmp_2017}\end{figure}

\begin{figure}[H]\begin{center}\adjustimage{max size={0.9\linewidth}{0.9\paperheight},height=0.25\paperheight}{Uganda_logit_model_files/output_103_0.png}\end{center}\caption{Timeseries of the drought score (blue) and normalized \(\Delta\)DMP
(orange) for ``ABIM'' district.}\label{fig_dmp_abim}\end{figure}

\begin{figure}[H]\begin{center}\adjustimage{max size={0.9\linewidth}{0.9\paperheight},height=0.25\paperheight}{Uganda_logit_model_files/output_104_0.png}\end{center}\caption{Correlation coefficients of the normalized \(\Delta\)DMP with the
drought score across Uganda districts.}\label{fig_dmp_corr_districts}\end{figure}

\begin{figure}[H]\begin{center}\adjustimage{max size={0.9\linewidth}{0.9\paperheight},height=0.25\paperheight}{Uganda_logit_model_files/output_105_0.png}\end{center}\caption{Distribution of the correlation coefficients of the normalized
\(\Delta\)DMP with the drought score across Uganda districts (related to
Figure \ref{fig_dmp_corr_districts}).}\label{fig_dmp_corr_distribution}\end{figure}

\begin{figure}[H]
\hypertarget{fig_ugandamap}{%
\begin{center}
\adjustimage{max size={0.9\linewidth}{0.9\paperheight},width=0.45\linewidth}{./pictures/Ugandamap.png}
\end{center}
\caption{Map of Uganda (related to Figure \ref{fig_dmp_corr_districts}).}\label{fig_ugandamap}
}
\end{figure}

\hypertarget{resources}{%
\section{Resources}\label{resources}}

This report has been generated from a
\href{https://github.com/rodekruis/IBF-system/blob/master/trigger-model-development/drought/skill-assessment/Model\%20Event\%20Data/Uganda_logit_model.ipynb}{Jupyter
notebook}. All the associated
\href{https://github.com/rodekruis/IBF-system/tree/master/trigger-model-development/drought/skill-assessment/Model\%20Event\%20Data/datasets}{datasets},
\href{https://github.com/rodekruis/IBF-system/blob/master/trigger-model-development/drought/skill-assessment/Model\%20Event\%20Data/drought_ibf_utility.py}{python
scrips} and the
\href{https://github.com/rodekruis/IBF-system/blob/master/trigger-model-development/drought/skill-assessment/Model\%20Event\%20Data/converted/Uganda_logit_model.tex}{\LaTeX source
file} of this report can be accessed via the
\href{https://github.com/rodekruis/IBF-system/tree/master/trigger-model-development/drought/skill-assessment/Model\%20Event\%20Data}{Rodekruis
GitHub}.

\end{document}
